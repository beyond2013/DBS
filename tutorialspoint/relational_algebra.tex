\PassOptionsToPackage{unicode=true}{hyperref} % options for packages loaded elsewhere
\PassOptionsToPackage{hyphens}{url}
%
\documentclass[]{article}
\usepackage{lmodern}
\usepackage{amssymb,amsmath}
\usepackage{ifxetex,ifluatex}
\usepackage{fixltx2e} % provides \textsubscript
\ifnum 0\ifxetex 1\fi\ifluatex 1\fi=0 % if pdftex
  \usepackage[T1]{fontenc}
  \usepackage[utf8]{inputenc}
  \usepackage{textcomp} % provides euro and other symbols
\else % if luatex or xelatex
  \usepackage{unicode-math}
  \defaultfontfeatures{Ligatures=TeX,Scale=MatchLowercase}
\fi
% use upquote if available, for straight quotes in verbatim environments
\IfFileExists{upquote.sty}{\usepackage{upquote}}{}
% use microtype if available
\IfFileExists{microtype.sty}{%
\usepackage[]{microtype}
\UseMicrotypeSet[protrusion]{basicmath} % disable protrusion for tt fonts
}{}
\IfFileExists{parskip.sty}{%
\usepackage{parskip}
}{% else
\setlength{\parindent}{0pt}
\setlength{\parskip}{6pt plus 2pt minus 1pt}
}
\usepackage{hyperref}
\hypersetup{
            pdfborder={0 0 0},
            breaklinks=true}
\urlstyle{same}  % don't use monospace font for urls
\setlength{\emergencystretch}{3em}  % prevent overfull lines
\providecommand{\tightlist}{%
  \setlength{\itemsep}{0pt}\setlength{\parskip}{0pt}}
\setcounter{secnumdepth}{0}
% Redefines (sub)paragraphs to behave more like sections
\ifx\paragraph\undefined\else
\let\oldparagraph\paragraph
\renewcommand{\paragraph}[1]{\oldparagraph{#1}\mbox{}}
\fi
\ifx\subparagraph\undefined\else
\let\oldsubparagraph\subparagraph
\renewcommand{\subparagraph}[1]{\oldsubparagraph{#1}\mbox{}}
\fi

% set default figure placement to htbp
\makeatletter
\def\fps@figure{htbp}
\makeatother


\date{}

\begin{document}

Relational database systems are expected to be equipped with a query
language that can assist its users to query the database instances.
There are two kinds of query languages − relational algebra and
relational calculus.

\hypertarget{relational-algebra}{%
\section{Relational Algebra}\label{relational-algebra}}

Relational algebra is a procedural query language, which takes instances
of relations as input and yields instances of relations as output. It
uses operators to perform queries. An operator can be either unary or
binary. They accept relations as their input and yield relations as
their output. Relational algebra is performed recursively on a relation
and intermediate results are also considered relations.

The fundamental operations of relational algebra are as follows −

\begin{itemize}
\tightlist
\item
  Select
\item
  Project
\item
  Union
\item
  Set different
\item
  Cartesian product
\item
  Rename
\end{itemize}

We will discuss all these operations in the following sections.

\hypertarget{select-operation-ux3c3}{%
\subsection{Select Operation (σ)}\label{select-operation-ux3c3}}

It selects tuples that satisfy the given predicate from a relation.

\textbf{Notation} − σp(r)

Where \textbf{σ} stands for selection predicate and r stands for
relation. p is prepositional logic formula which may use connectors like
\textbf{and}, \textbf{or}, and \textbf{not}. These terms may use
relational operators like − =, ≠, ≥, \textless{} , \textgreater{}, ≤.

\textbf{For example} −

\begin{verbatim}
σsubject = "database"(Books)
\end{verbatim}

\textbf{Output} − Selects tuples from books where subject is `database'.

\begin{verbatim}
σsubject = "database" and price = "450"(Books)
\end{verbatim}

\textbf{Output} − Selects tuples from books where subject is `database'
and `price' is 450.

\begin{verbatim}
σsubject = "database" and price = "450" or year > "2010"(Books)
\end{verbatim}

\textbf{Output} − Selects tuples from books where subject is `database'
and `price' is 450 or those books published after 2010.

\hypertarget{project-operation}{%
\subsection{Project Operation (∏)}\label{project-operation}}

It projects column(s) that satisfy a given predicate.

Notation − ∏A1, A2, An (r)

Where A1, A2 , An are attribute names of relation r.

Duplicate rows are automatically eliminated, as relation is a set.

\textbf{For example} −

\begin{verbatim}
∏subject, author (Books)
\end{verbatim}

Selects and projects columns named as subject and author from the
relation Books.

\hypertarget{union-operation}{%
\subsection{Union Operation (∪)}\label{union-operation}}

It performs binary union between two given relations and is defined as −

\begin{verbatim}
r ∪ s = { t | t ∈ r or t ∈ s}
\end{verbatim}

\textbf{Notation} − r U s

Where \textbf{r} and \textbf{s} are either database relations or
relation result set (temporary relation).

For a union operation to be valid, the following conditions must hold −

r, and s must have the same number of attributes. Attribute domains must
be compatible. Duplicate tuples are automatically eliminated.

\begin{verbatim}
∏~author~ (Books) ∪ ∏ author (Articles)
\end{verbatim}

\textbf{Output} − Projects the names of the authors who have either
written a book or an article or both.

\hypertarget{set-difference}{%
\subsection{Set Difference (−)}\label{set-difference}}

The result of set difference operation is tuples, which are present in
one relation but are not in the second relation.

\textbf{Notation − r − s}

Finds all the tuples that are present in \textbf{r} but not in
\textbf{s}.

\begin{verbatim}
∏~author~ (Books) − ∏~author~ (Articles)
\end{verbatim}

\textbf{Output} − Provides the name of authors who have written books
but not articles.

\hypertarget{cartesian-product-ux3c7}{%
\subsection{Cartesian Product (Χ)}\label{cartesian-product-ux3c7}}

Combines information of two different relations into one.

\textbf{Notation} − r Χ s

Where \textbf{r} and \textbf{s} are relations and their output will be
defined as −

\(r Χ s = { q t | q ∈ r and t ∈ s}\)

\begin{verbatim}
σ~author~ = 'tutorialspoint'(Books Χ Articles)
\end{verbatim}

\textbf{Output} − Yields a relation, which shows all the books and
articles written by tutorialspoint.

\hypertarget{rename-operation-ux3c1}{%
\subsection{Rename Operation (ρ)}\label{rename-operation-ux3c1}}

The results of relational algebra are also relations but without any
name. The rename operation allows us to rename the output relation.
`rename' operation is denoted with small Greek letter \textbf{rho}
\emph{ρ}.

\textbf{Notation} − ρ x (E)

Where the result of expression \textbf{E} is saved with name of
\textbf{x}.

Additional operations are −

\begin{itemize}
\tightlist
\item
  Set intersection
\item
  Assignment
\item
  Natural join
\end{itemize}

\hypertarget{relational-calculus}{%
\subsection{Relational Calculus}\label{relational-calculus}}

In contrast to Relational Algebra, Relational Calculus is a
non-procedural query language, that is, it tells what to do but never
explains how to do it.

Relational calculus exists in two forms −

\hypertarget{tuple-relational-calculus-trc}{%
\subsection{Tuple Relational Calculus
(TRC)}\label{tuple-relational-calculus-trc}}

Filtering variable ranges over tuples

\textbf{Notation} − \{T \textbar{} Condition\}

Returns all tuples T that satisfies a condition.

\textbf{For example} −

\({ T.name | Author(T) AND T.article = 'database' }\)

\textbf{Output} − Returns tuples with `name' from Author who has written
article on `database'.

TRC can be quantified. We can use Existential (∃) and Universal
Quantifiers (∀).

\textbf{For example} −

\({ R| ∃T ∈ Authors(T.article='database' AND R.name=T.name)}\)

\textbf{Output} − The above query will yield the same result as the
previous one.

\hypertarget{domain-relational-calculus-drc}{%
\subsection{Domain Relational Calculus
(DRC)}\label{domain-relational-calculus-drc}}

In DRC, the filtering variable uses the domain of attributes instead of
entire tuple values (as done in TRC, mentioned above).

\textbf{Notation} −

\({ a1, a2, a3, ..., an | P (a1, a2, a3, ... ,an)}\)

Where a1, a2 are attributes and \textbf{P} stands for formulae built by
inner attributes.

\textbf{For example} −

\({< article, page, subject > | ∈ TutorialsPoint ∧ subject = 'database'}\)

\textbf{Output} − Yields Article, Page, and Subject from the relation
TutorialsPoint, where subject is database.

Just like TRC, DRC can also be written using existential and universal
quantifiers. DRC also involves relational operators.

The expression power of Tuple Relation Calculus and Domain Relation
Calculus is equivalent to Relational Algebra.

\end{document}
